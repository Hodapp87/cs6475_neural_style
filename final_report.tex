% Chris Hodapp, 2016-04-25
% Georgia Institute of Technology, CS6475, Computational Photography,
% Spring 2016
% Final Project

\documentclass{article}
\usepackage{amsmath}
\usepackage{graphicx}
\usepackage{float}
\usepackage{hyperref}
\usepackage[letterpaper, portrait, margin=1in]{geometry}

\title
    {CS6475, Computational Photography \\
      Spring 2016 \\
      Final Project}
\date{April 25, 2016}
\author{Chris Hodapp, chodapp3@gatech.edu}

\begin{document}

\section{Literate Implementation of ``A Neural Algorithm of Artistic Style''}

\subsection{Project Goal}

The goal of this project was to implement the algorithm described in
the paper.

The eventual result was a Python notebook (for Jupyter/IPython) which
gives a simplified example of how to actually implement this
algorithm.  It aims to use most of the same libraries that we used in
the class where possible, with caffe being the main addition.

I found that many of the existing implementations lacked clear
explanations for why they were written the way they did.  I wanted to
create an implementation that other students could refer to and
hopefully understand.

\subsection{Images}

\subsection{Pipeline}

\subsection{Project Resources}

% What is the best way to see your project?
% Provide links, pictures, describe it, give details.

\subsection{What Worked \& What Didn't}

- The algorithm could be very finicky about textures, and did not seem
to handle high-detail photos very well.

- However, there is room for a lot of tuning.  The layers chosen have
a lot of influence over this.  The parameters seemed very
scale-dependent too.

- It seemed as though the style image needed to have a sufficient
range of color and texture in it in order to be able to represent the
content well.

- Poor settings seemed to manifest themselves as the content image
being completely unrecognizable, or as the content image being visible
as a mosaic of the same pieces of the style image over and over.

- I quickly ran into the memory and processing limits.  My video card
was an Nvidia GTX 660 Ti with 2 GB of memory, and while some neural
networks fit in this memory, many others (the preferred ones) could
not.  In addition, it put a limit on how large of an image I could
generate.  While I could have run all of these operations on my CPU,
this would have taken between hours and days for a single image.

\subsection{Other Details}

\subsection{References/Pointers}

% Reference:
http://jessicalanan.com/wp-content/uploads/2013/09/Fall-FB.jpg

\subsection{Team}

\subsection{Credits/Thanks}


% Include PDF:
% \includegraphics{ps1-002}

\end{document}
