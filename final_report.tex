% Chris Hodapp, 2016-04-25
% Georgia Institute of Technology, CS6475, Computational Photography,
% Spring 2016
% Final Project

\documentclass{article}
\usepackage{amsmath}
\usepackage{graphicx}
\usepackage{float}
\usepackage{hyperref}
\usepackage[letterpaper, portrait, margin=1in]{geometry}

\title
    {CS6475, Computational Photography \\
      Spring 2016 \\
      Final Project}
\date{April 25, 2016}
\author{Chris Hodapp, chodapp3@gatech.edu}

\begin{document}

\section{Literate Implementation of ``A Neural Algorithm of Artistic Style''}

\subsection{Project Goal}

The goal of this project was to create a literate implementation (as
in \href{https://en.wikipedia.org/wiki/Literate_programming}{literate
  programming}) of the algorithm described in the recent paper, ``A
Neural Algorithm of Artistic Style'' by Leon A. Gatys, Alexander
S. Ecker, and Matthias Bethge\cite{neuralstyle2015}.

The paper is sufficiently well-known by now that it has many
open-source and commercial implementations that are quite good:

\begin{itemize}
  \item \url{https://github.com/jcjohnson/neural-style}
  \item \url{https://github.com/kaishengtai/neuralart}
  \item \url{https://github.com/andersbll/neural_artistic_style}
  \item \url{https://github.com/fzliu/style-transfer}
  \item \url{https://github.com/woodrush/neural-art-tf}
  \item \url{https://deepart.io}
\end{itemize}

However, I found that many of the existing implementations lacked
clear explanations for why they were written the way they did.

The students in this class (and others such as CS6476) should already
be familiar to some degree with the use of Python, NumPy, SciPy, and
OpenCV in the algorithms of computational photography.  My aim was
that this implementation let such students use this familiarity as a
stepping stone to understanding the clever methods in this paper and
libraries like the \href{http://caffe.berkeleyvision.org/}{Caffe} deep
learning framework.

The eventual result was an IPython notebook (via
\href{https://jupyter.org/}{Jupyter}) which gives a simplified (but
still functional) example of how to actually implement this algorithm.
All images in this report were produced with that code.

% TODO: And where is that code?

\subsection{Images}

\subsection{Pipeline}

\subsection{Project Resources}

% TODO: What is the best way to see your project?
% TODO: Provide links, pictures, describe it, give details.

\subsection{What Worked \& What Didn't}

- The algorithm could be very finicky about textures, and did not seem
to handle high-detail photos very well.

- However, there is room for a lot of tuning.  The layers chosen have
a lot of influence over this.  The parameters seemed very
scale-dependent too.  Tuning unfortunately tended to be difficult when
the results were subject to some amount of randomness.

- It seemed as though the style image needed to have a sufficient
range of color and texture in it in order to be able to represent the
content well.

- Poor settings seemed to manifest themselves as the content image
being completely unrecognizable, or as the content image being visible
as a mosaic of the same pieces of the style image over and over.

- I quickly ran into the memory and processing limits.  My video card
was an Nvidia GTX 660 Ti with 2 GB of memory, and while some neural
networks fit in this memory, many others (the preferred ones) could
not.  In addition, it put a limit on how large of an image I could
generate.  While I could have run all of these operations on my CPU,
this would have taken between hours and days for a single image.

\subsection{Other Details}

\subsection{References/Pointers}

% Reference:
http://jessicalanan.com/wp-content/uploads/2013/09/Fall-FB.jpg
http://jessicalanan.com/an-autumn-stroll/
% and all the other stuff I linked
   
\subsection{Team}

\subsection{Credits/Thanks}

% Do links up above belong here?
% Chris Harmoney?

\begin{thebibliography}{9}

\bibitem{neuralstyle2015} Leon A. Gatys, Alexander S. Ecker, and
  Matthias Bethge, A Neural Algorithm of Artistic Style,
  \emph{arXiv:1508.06576v2}, \url{http://arxiv.org/abs/1508.06576},
  2015.

\end{thebibliography}

\end{document}
